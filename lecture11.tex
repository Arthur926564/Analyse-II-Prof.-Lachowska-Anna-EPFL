\lecture{11}{2025-03-24}{Differentiable}{}

\begin{parag}{Méthode $7$ Récurrence}
    \begin{subparag}{Le principe fondamental de récurrence}
        Soit $S \subset \mathbb{N}$ sous-ensemble : $ 0 \in S$ et pour tout $n \in S$ on a $(n+1) \in S$. Alors $S = \mathbb{N}$
    \end{subparag}
    \begin{subparag}{Méthode de récurrence}
        Soit $P(n)$ une proposition qui dépend de $n \in \mathbb{N}$, $n \geq n_0$\\
Supposons que
        \begin{itemize}
            \item $P(n_0)$ est vraie
            \item $P(n)$ implique $P(n+1)$ pour tout $n \geq n_0$ naturel
        \end{itemize}
        Alors $P(n)$ est vraie pour tout $n \geq n_0$
    \end{subparag}
On regroupe quatre étapes pour une preuve par récurrence:
\begin{enumerate}
\item La proposition ( Soit $P(n)$ la proposition pour $x$)
\item L'initialisation $P(0)$
\item L'hérédité: Supposons que $P(n)$ est vrai, alors il faut en déduire $P(n+1)$
\item Conclusion: Puisque $P(x_0)$ est vraie et que pour tout $x \geq x_0$, $P(n) \implies P(n+1)$, par récurrence $P(n)$ est vraie $ \forall n \geq x_0$. 
    
\end{enumerate}
\begin{framedremark}
    Attention a ne pas mélanger ce qu'on veut et ce qu'on a.
\end{framedremark}

\end{parag}
\begin{parag}{Récurrence généralisée}
    Soit $P(n)$ une proposition qui dépend de $n \in \mathbb{N}, n \geq n_0$.\\
    Supposons que $(1) P(n_0), \dots P(n_0 + k)$ sont vraie pour un $k \in \mathbb{N}$\\
    En deuxième $\{ P(n), P(n+1), \dots, P(n+k)\}$ impliquent $P(n +k+1) \forall n \geq n_0$, $n \in , \mathbb{N}$.\\
    Alors, $P(n)$ est vraie $ \forall n \geq n_0$, $ n \in \mathbb{N}$
\end{parag}

\chapter{Calcul différentielle des fonctions de plusieurs variables}

    \section{Dérivées parielles, le gradient}
    \begin{parag}{Dérivée partielle}
        \begin{definition}
            Soit $f : E \to \mathbb{R}$ une fonction, $E \subset \mathbb{R}$ sous-ensemble ouvert. \\
        Soit $g(s) = f(a_1, a_2, \dots, \overbrace{s}^{k}, a_{k+1}, \dots, a_n)$ où $ \overline{a} = (a_1, \dots, a_n) \in E$.
        \begin{align*}
            g: D = \{ s \in \mathbb{R} : (a_1, a_2, \dots, s, a_{k+1}, \dots, a_n) \in E\} \to R
        \end{align*}
        Alors si $g$ est dérivable en $a_k \in D$, on dit que la \important{k-ième dérivée partielle} de $f$ en $ \overline{a} \in E$ existe et est égale à $g'(a_k)$
        \begin{framedremark}
            Notation: $ \frac{ \partial f}{ \partial x_k}( \overline{a}) \equiv D_k f( \overline{a})$
        \end{framedremark}
        \end{definition}
        
       On a:
        \begin{align*}
            \frac{ \partial f}{ \partial x_k} ( \overline{a}) = \lim_{ t \to 0} \frac{g(a_k + t) - g(a_k)}{t} = \lim_{t \to 0} \frac{f( \overline{a} + t \overline{e_k}) - f( \overline{a})}{t}
        \end{align*}
      
    \end{parag}
    
   \begin{parag}{Gradient}
       \begin{definition}
           Si toutes les dérivées partielles existent en $ \overline{a} \in E$: $ \frac{ \partial f}{ \partial x}( \overline{a})  \dots \frac{ \partial f}{ \partial x_n}( \overline{a})$, alors on définit le \important{gradient} de $f$ en $ \overline{a}$ comme:
           \begin{align*}
             \nabla  f( \overline{a}) = \left( \frac{ \partial f}{ \partial x}( \overline{a}), \dots \frac{ \partial f}{ \partial x_2}( \overline{a}) , \dots\frac{ \partial f}{ \partial x_n}( \overline{a}) \right) 
           \end{align*}
           
       \end{definition}
   
   \end{parag}
    
       \section{Dérivée directionnelle}
       \begin{parag}{Définition}
           Soit $E \subset \mathbb{R}^n $ sous-ensemble ouvert, $ \overline{a} \in E$, $\overline{v} \in \mathbb{R}^n , \overline{v} \neq 0$ La droite passant par $ \overline{a}$ en direction $ \overline{v}$ admet la paramétrisation $ \overline{e}(t) = \overline{a}  + t \overline{v}$ et cela $ \forall t \in \mathbb{R}$.\\
           Considérons la fonction $ f : E \to \mathbb{R}$\\
           et soit $g(t) = f( \overline{a} + t + \overline{v})$ la fonction d'une seule variable $t \in \mathbb{R}$:
           \begin{align*}
               g: D = \{t \in \mathbb{R}: \overline{a} + t \overline{v} \in E\} \to R
           \end{align*}
           \begin{definition}
               Si $g$ est dérivable en $t = 0$ on dit qu'il existe \important{la dérivée directionnelle} de $f$ en $ \overline{a}$ suivant le vecteur $ \overline{v}$ (en direction de $\overline{v}$)\\
               La dérivée directionnelle de $f$ en $ \overline{a}$ en direction de $ \overline{v}$ est:
               \begin{align*}
                   Df( \overline{a}, \overline{v}) = \frac{\partial f}{\partial \overline{v}}( \overline{a}) = \lim_{t \to 0} \frac{g(t) - g(0)}{t} = \lim_{t \to 0} \frac{f( \overline{a} + t \overline{v}- f( \overline{a}}{t}
               \end{align*}
               
           \end{definition}
           
       \begin{framedremark}
           Si $ \overline{v} = \overline{e}_i$ ou $ \overline{e_i}$ est un vecteur unitaire, Alors \begin{align*}
              Df( \overline{a}, \overline{e}_i) = \frac{\partial f}{\partial x_i}( \overline{a})
           \end{align*}
           Si toutes les dérivées directionnelles existent en $ \overline{a}$ (pour tout $ \overline{v} \neq \overline{0}$), alors toutes les dérivées partielles existent en $ \overline{a}$. La réciproque est fausse en générale
       \end{framedremark}
         \begin{framedremark}
             \begin{align*}
                 Df( \overline{a}, \lambda \overline{v}) = \lambda \cdot Df( \overline{a}, \overline{v}) \;\; \forall \lambda \in \mathbb{R}: \; \lambda \neq 0
             \end{align*}
             
         \end{framedremark}
           
       
       \end{parag}
       
       
           \section{Dérivabillité et la différentielle}
           \begin{definition}
               Soit $f: E \to \mathbb{R}$, $E \subset \mathbb{R}^n $ ouvert, $ \overline{a} \in E$\\
               On dit que $f$ est \important{dérivable} au point $ \overline{a}$ s'il existe une transformation linéaire:
               \begin{align*}
                   L_{ \overline{a}}: \mathbb{R}^n  \to \mathbb{R}
               \end{align*}
               et une fonction $ r: E \to \mathbb{R}$ telle que:
               \begin{align*}
                   f(x) = f( \overline{a}) + L_{ \overline{a}}( \overline{x} - \overline{a}) + r ( \overline{x}) \; \; \forall \overline{x} \in E\\
                   \lim_{ \overline{x} \to \overline{a}} \frac{r( \overline{x})}{ \mid \mid \overline{x} - \overline{a} \mid \mid}
               \end{align*}
           \end{definition}
           \begin{definition}
               $L_{ \overline{a}}$ s'appelle \important{la différentielle} de $f$ au point $ \overline{a} \in E$\\
Notation:
\begin{align*}
    L_{ \overline{a}} = df( \overline{a})
\end{align*}

           \end{definition}
           \begin{framedremark}
               Une transformation linéaire $ T: \mathbb{R}^n  \to \mathbb{R}$ est une fonction telle que $\tau ( \alpha \overline{x}_1 + \beta \overline{x}_2) = \alpha T( \overline{x}_1) + \beta T( \overline{x}_2)$ pour tout $ \overline{x}_1, \overline{x}_2 \in \mathbb{R}^n $
Par example $x + 3y$ est une transformation linéaire tandis que $ x + 2y + 2$ n'en est pas une.
               
           \end{framedremark}
           
           
           
