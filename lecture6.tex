\lecture{6}{2025-03-06}{EDL2}{}
\subsection{Méthode de résolution de EDL2}

\begin{parag}{Rappel (Méthode de la variation des constantes)}
    En premier lieu on calcule le Wronkien de $v_1(x)$ et $v_2(x)$, 
    \begin{align*}
        W[v_1, v_2] = \det \begin{pmatrix}
            v_1  & v_2 \\
            v_1' & v_2'
        \end{pmatrix}
    \end{align*}
    Ensuite, On calcule les fonctions $c_1(x)$ et $c_2(x)$:
    \begin{align*}
        c_1(x) = -\int \frac{f(x)v_2(x)}{W[v_1, v_2]}dx \\
        c_2(x) = \int \frac{f(x)v_1(x)}{W[v_1, v_2]}dx
    \end{align*}

    
    

\end{parag}

\begin{parag}{Méthode de calcul}
    Pour des fonctions $f(x)$ spéciales, une méthode alternative existe:
    \begin{subparag}{Case 1}
        si $f(x)$ est de la forme:
        \begin{align*}
            f(x) = e^{cr}R_n(x)
        \end{align*}
        avec $R_n(x)$ un polynôme de degré $n \in \mathbb{N}_{ \geq 0}$.
       \\
       Alors la solution est donné par:
       \begin{align*}
           \implies y_p(x) = x^r e^{cx}T_n(x)
       \end{align*}
       Avec $r = 0, 1$ ou $2$ la multiplicité de la racine $c$ dans l'équation caractéristique, \important{$T_n(x)$} un polynôme à déterminer de degré $n$.
    \end{subparag}
    \begin{subparag}{Cas 2}
        \begin{align*}
            f(x) = e^{ \alpha x}( \cos ( \beta x) P_k(x) + \sin( \beta x) Q_n(x)); \; \; \alpha, \beta \in \mathbb{R} \\
            \implies y_p(x) = x^r e^{ \alpha x} ( \cos( \beta x ) T_n(x) + \sin ( \beta x ) S_n(x))
        \end{align*}
        Avec $r = 1$ si $ \alpha + i \beta$ est racine de l'équation caractéristique, et $r = 0$ sinon, $T_n(x)$ et $S_n(x)$ des polynômes à déterminer de degré $n = \text{ max}(k, m)$
\end{subparag}
    Pour déterminer les \important{coefficients des polynômes inconnus}:
    \begin{itemize}
        \item Calculer les dérivées de la solution particulière
        \item Remplacer dans l'équation initiale, et résoudre l'équation.
    \end{itemize}
\end{parag}
    \begin{parag}{Exemple}
        \begin{align*}
            y'' + 2y' + 10y = 40 e^{x}\sin(3x)
        \end{align*}
        \textbf{Solution homogène:}
        \\
        \begin{align*}
            \lambda^2 + 2 \lambda + 10 = 0\\
            \lambda_{1, 2} = \frac{-2 \pm \sqrt{4 - 40}}{2} = \frac{-2 \pm i6}{2}\\
            = -1 \pm i3
        \end{align*}
        On cherche maintenant $y_h(x)$ :
        \begin{align*}
            y_h(x) = C_1 e^{-2x} \cos(3x) + C_2e^{-x^2}\sin(3x)
        \end{align*}
        \textbf{Coefficient indeterminé}
        \\
        $f(x) = 40 \cdot e^x \sin(3x)$, on cherche doncune fonction qui rempli ce critère:
        \begin{align*}
           f(x)e^{ \alpha x}(\cos( \beta x) \underbrace{P_k(x)}_{=A} + \sin( \beta x) \underbrace{Q_m(x)}_{=B})
        \end{align*}
        On sait que $ \beta = 3$ et que $ \alpha = 1$:
        \begin{framedremark}
           Il n'y a pas de rapport direct entre ce $ \alpha  =1$ et la solution de l'équation caractéristique.
           \\
           On observe la fonction $f(x)$ qui ici à pour l'exponentielle $e^x = e^{ 1 \cdot x}$, c'est de là que vient notre $ \alpha$
        \end{framedremark}
        
        \begin{align*}
          \alpha + i \beta = 1 + i3 \implies r = 0
        \end{align*}
        Comme $r = 0$ on sait donc que le polynôme n'est qu'une constante qu'on va noter $T_n(x) = A$ et $S_n(x) = B$\\
        On peut noter donc notre fonction pour laquelle on cherche les coefficients:
        \begin{align*}
            y_p(x) = e^3(\cos(3x)A + \sin( 3x)B)
        \end{align*}
        On va dérivée tout ce beau monde:
        \begin{align*}
            y_p'(x) &= e^x(\cos(3x)A + \sin(3x)B - r\sin(3x)A + 3\cos(3x)B) \\
            y_p''(x) &= e^x(\cos(3x)(A + 3B) + \sin(3x)( B - 3A) - 3(A + 3B)\sin(3x) + 3\cos(3x)(B - 3A)) \\
                     &= e^x( \cos(3x)( 6B -8A) + \sin(3x)(-8B - 6A))
        \end{align*}
        On injecte tout ca dans l'EDL2:
        \\
        On peut tout diviser par $e^x$ car il se trouve partout et n'est jamais égal à $0$. \\
        \begin{align*}
            \cos(3x)(6B - 8A + 2(A + 3B) + 10A) + \sin(3x)(-8B - 6A)\\
            + 2(B - 3A) + 10B) = 40 \sin(3x)
        \end{align*}
        On voit ici que tout la partie du $\cos(x)$ est égal à $0$, c'est comme ci on avait deux équation, la partie avec le $\cos(x)$ et la partie avec le $\sin(x)$:
        \begin{align*}
          \implies  \begin{cases}
                12B + 4A = 0 \\
                4B -12A = 40
            \end{cases}
            \implies \begin{cases}
                A = -3B \\
                4B + 36B = 40
            \end{cases} \implies
            \begin{cases}
                A = -3 \\ B= 1
            \end{cases}
        \end{align*}
       On obtient donc que la solution particulière:
       \begin{align*}
           y_p(x) = e^x(-3\cos(3x) + \sin(3x))
       \end{align*}
       Et pour la solution générale:
       \begin{align*}
           y = C_1e^{-x}\cos(3x) + C_2e^{-x^2}\sin(3x) + e^x(-3\cos(3x) + \sin(3x))
       \end{align*}
    \end{parag}
    \begin{parag}{Exemple}
        \begin{align*}
            y'' + 2y' - 3y = (x + 1)e^{-3x}
        \end{align*}
        Ici nous somme dans le cas numéro $1$:
        \begin{align*}
            f(x) = e^{cx}R_n(x) \implies c = -3, n = 1
        \end{align*}
       \begin{align*}
        y_p(x) = x^re^{-3x}(Ax + B) \text{ où } r = 1 \\
        y_p(x) = e^{-3x}(Ax^+ Bx)
       \end{align*}
       Ici on a pris un polynôme $R_n$ de puissance $1$ c'est pour cela qu'on peut l'écrire comme nous l'avons fait ci-dessus.
       \\
       Donc ici on va dérivée $y_p$ deux fois et tout remettre dans l'équation de base et ensuite résoudre:
       \begin{align*}
           y_p'(x) &= e^{-3x}(-3(Ax^2 + Bx) + 2Ax + B) = e^{-3x}(-3Ax^2 + (2A -3B)x + B) \\
           y_p''(x) &= e^{-3x}(9Ax^2 +  (-6A + 9B)x - 3B + (-6A)x + (2A - 3B)) \\
                    &= ^{-3x}(9A + (-12A + 9B)x + 2A - 6B)
       \end{align*}
       On divide l'EDL2 par $e^{-3x}$ ce qui nous donne:
       \begin{align*}
           &9Ax^2 + (-12A + 9B)x + 2A - 6B + 2(-3Ax^2 + (2A - 3B)x + B) -3Ax^2 + Bx = x+1 \\
           &\implies \begin{cases}
               (9A - 6A -3A)x^2 = 0x^2 \\
               (-2A + 9B + 9A - 6B - 3B)x = x \\
               (2A - 6B) + 2B = 1
           \end{cases} \\
           &\implies \begin{cases}
               -8A = 1 \\
               2A - 4B = 1
           \end{cases} \implies \begin{cases}
               A = - \frac{1}{8}\\
               B = - \frac{5}{16}
           \end{cases}
       \end{align*}
       
       On obtient finalement pour la solution particulière:
       \begin{align*}
           y_p(x) = e^{-3x} \left( - \frac{x^2}{8} - \frac{5x}{16} \right) \\
          y_h(x) = C_1 e^{-3x} + C_2e^x \; \; \; C_1, C_2 \in \mathbb{R}
       \end{align*}
       
       

        
        
        
 

\end{parag}

\subsection{Méthode de démonstration par l'absurde}
\begin{parag}{Méthode}
    On a une relation tel que:
    \begin{align*}
        T \implies Q \equiv \neg Q \implies F
    \end{align*}
    \begin{subparag}{Exemple}
        $\neg \exists \in \mathbb{Z} \text{ tel que } 18x - 54 y = 21$:
        \\
        Supposons $\neg Q$ tel qu'il existe $x, y$ tel que:
        \begin{align*}
            18x - 54 y = 21 \\
            \underbrace{2x - 6y}_{ \in \mathbb{Z} = \frac{21}{9} = \underbrace{ \frac{7}{3}}_{ \notin \mathbb{Z}}
        \end{align*}
        Comme cela est impossible, alors la il ne peux exister de $x, y \in \mathbb{Z}$ tel que la relation tienne:
        \begin{align*}
            \neg Q \implies \neg P
        \end{align*}
    \end{subparag}
    
\end{parag}

      \begin{parag}{Euclide}
        \begin{theoreme}
           soit $ \mathbb{P}$ l'ensemble des nombres premiers alors:
           \begin{align*}
               \mid \mathbb{P} \mid = \infty
           \end{align*}
        \end{theoreme}
        \textbf{Démonstration} \\
        Supposons $ \exists n \in \mathbb{N} \text{ tel que } \mid \mathbb{P} \mid = n < \infty$ 
        \begin{align*}
        \mathbb{P} = \à p_1, \dots, p_n\}
        \end{align*}
        Soit $k = p_1  \cdot p_2 \cdot\dots \cdot p_{n} + 1$,  alors, $k \notin \mathbb{P}$ car $k > p_i \forall i = i \dots n$ (Il ne peux pas être dans $ \mathbb{P}$ car il est plus grand que tout les éléments de $ \mathbb{P}$) \\
        Et donc, il existe un éléments dans $ \mathbb{P}$ tel qu'il divise $k$:
        \begin{align*}
            \exists p_j \neq k \text{ tel que } p_j \mid k
        \end{align*}
       Si on note
       \begin{align*}
           \underbrace{k - p_1 \cdot p_2 \cdot \dots \cdot p_n}_{ \text{divisible par} p_j} = \underbrace{1}_{ \text{pas div}}
       \end{align*}
       Ce qui est une contradiction logique.
       \begin{framedremark}
           la partie $k - p_1p_2 \dots p_n$ est divisible par $p_j$ car de $(1)$ on n'a dit que $k$ l'était ( juste au dessus) et le produit de tout les $p_n$ est forcement divisible par $p_j$ vu que $p_j \in \mathbb{P}$. L'addition de deux nombre divisible par un nombre est forcement divisible par ce dernier.
       \end{framedremark}
       
        
        
      \end{parag}
      










