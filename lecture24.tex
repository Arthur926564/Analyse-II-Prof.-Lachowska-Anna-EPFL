\lecture{24}{2025-05-14}{Intégrale sur un domaine borné}{}
\begin{parag}{Example avec le théorème de Fubini}
    Le but ici est de calculer le volume du sous-ensembe de  $\mathbb{R}^{3}$ défini par:
    \begin{align*} \left\{\left(x, y, z\right) in \mathbb{R}^{3}: 0 \leq x \leq 4, 0 \leq y \leq 3, 0 \leq z \leq \left( 1 + 3x + x\sin\left(xy\right)\right)\right\} \end{align*}
    \begin{framedremark}
        La fonction $1 + x\left(3 + \sin\left(xy\right)\right) > 0$ et elle est continue.
    \end{framedremark}
    Donc on a que $P =  \underbrace{\left[0, 4\right]}_{x} \times \underbrace{\left[0, 3\right]}_{y}$.\\
    On calcule dont le volume $V =  \int_\P\int \left(1 + 3x + x\sin\left(xy\right)\right)dxdy$ Qui par la linéarité nous donne:

    \begin{align*}  
        V &= \int_\P\int 1dxdy + 3\int\int_Pxdxdy + \int\int_Px\sin\left(xy\right)dxdy\\
          &= 1\cdot \overbrace{1\left(4-0\right)\left(3-0\right)}^{= 12} + 3 \int_0^5\left(\int_0^4 xdx\right)dy + \int\int_px\sin\left(xy\right)dxdy 
    \end{align*}
    Pour la deuxième intégrale on a:
    \begin{align*} 3\int_0^5\left(\int_0^4xdx\right)dy = 3\int_0^31dy\cdot \int_0^4xdx\\
    = 3y_{0 \to 3}\cdot \frac{1}{2}x^2_{0 \to 4} =  9 \cdot  8 =  72\end{align*}
    Pour la dernière intégrale on a:
    \begin{align*} \int\int_P x \sin\left(xy\right)dxdy =  \int_0^4\left(\int_0^3x\sin\left(xy\right)dy\right)dx =  \int_0^4\left(\int_0^3 \sin\left(xy\right)d\left(xy\right)\right)dx\\
    = \int_0^4 - \cos\left(xy\right)_{0 \to 3}dx\\
    = \int_0^4 \left(-\cos\left(3x\right) + 1 dx =  -\frac{1}{3}\sin\left(3x\right) + x_{0 \to 4} =  -\frac{1}{3}\sin\left(12\right) + 4
\end{align*}
Ce qui nous donne pour le volume final: $V =  12 + 72 - \frac{1}{3}\sin\left(12\right) =  88 - \frac{1}{3}\sin\left(12\right)$.
\end{parag}
\begin{parag}{Dans l'autre sens}
    On peut considérer la même intégrale  dans l'autre sens $\int_0^3\left(\int_0^4 x \sin\left(xy\right)dx\right)dy$\\
    Pour résoudre cela on va faire une intégration par partie ce qui nous donne:
    \begin{align*} 
        \int_0^4 x \sin\left(xy\right)dx =  -\int_0^4\frac{x}{y}d\left(\cos\left(xy\right)\right) 
    \end{align*}
    \begin{align*} 
        = - \frac{x\cos\left(xy\right)}{y}_{0\to 4} + \int_0^4 \cos\left(xy\right) \cdot  \frac{1}{y}dx\\
        = -\frac{4\cos\left(4y\right)}{y} + \frac{1}{y^2}\sin\left(xy\right)_{0 \to 4} = -\frac{4\cos\left(4y\right)}{y} + \frac{\sin\left(4y\right)}{y^2}
    \end{align*}
    Nous voyons que cette fonctions n'est pas définie en $0$ si l'on veut éviter de faire une intégrale impropre, ça peut être rentable d'essayer de prolonger la fonction en $y = 0$:\\
    Par un développement limité de sinus:
    \begin{align*} 
        \lim_{y  \to 0} \frac{\sin\left(4y\right) - 4y\cos\left(4y\right)}{y^2} =  \lim_{y \to 0} \frac{4y - \frac{1}{6}\left(4y\right)^3 + \ldots - 4y\left(1 - \frac{1}{2}\left(4y\right)^2^+ \ldots \right)}{y^2} \\
        = \lim_{y \to 0} \frac{C\left(y^3\right)}{y^2} =  0
    \end{align*}
    On a donc que:
    \begin{align*} 
        \int_0^3 \frac{\sin\left(4y\right) - 4y\cos\left(4y\right)}{y^2}dy = \int_0^3 \frac{\sin\left(4y\right)}{y^2}dy - \int_0^3 \frac{4y\cos\left(4y\right)}{y^2} dy
    \end{align*}
    Néanmoins ces primitives ne s'expriment pas en fonctions élémentaires. Pour quand même essayer de trouver un résultat convaincant, on peut en tirer des sommes:\\
L'idée ici c'est d'utiliser le faite que la dérivée de $\frac{1}{y}$ est $\frac{-1}{y^2}$:
    \begin{align*} 
        \int_0^3 \frac{\sin\left(4y\right)-4y\cos\left(4y\right)}{y^2} dy = \int_0^3\left(4y\cos\left(4y\right)-\sin\left(4y\right)d\left(\frac{1}{y}\right)\\
            = -\frac{\sin\left(4y\right)}{y} + 4\cos\left(4y\right)_{0 \to 3} - \int \frac{4\cos\left(4y\right) - 16y\sin\left(4y\right) - 4\cos\left(4x\right)}{y} dy\\
            = -\frac{\sin\left(12\right)}{3} + 4\cos\left(12\right) + \lim_{x \to 0} \frac{\sin\left(4y\right)}{y} - 4 + \int_0^3 16\sin\left(4y\right)dy\\
            = -\frac{\sin\left(12\right)}{3} + 4\cos\left(12\right) + 4 - 4 -4\cos\left(4y\right)_{0 \to 3}\\
            = -\frac{\sin\left(12\right)}{3} + 4\cos\left(12\right) - 4 \cos\left(12\right) + 4\\
            = 4- \frac{1}{3}\sin\left(12\right)
    \end{align*}
    Qui est le même résultat que l'intégrale précédente.
    
    \begin{subparag}{Conclusion}
        Il est important donc de choisir un bonne ordre pour l'intégration afin d'éviter des calculs techniques
    \end{subparag}
\end{parag}


\subsection{Intégrale sur un domaine bornée}
\begin{parag}{Définition}
    \begin{definition}
        Soit $E \subset P \subset \mathbb{R}^{n}$; $f: E \to \mathbb{R}$ fonction bornée sur $E$.\\
        Posons $\bhat{f}\left(\overline{x}\right) =  \begin{cases} f\left(\overline{x}\right), \overline{x} \in E\\ 0, \overline{x} \in P \setminus E \end{cases}$ La fonction $f$ est inégrable sur $E$ si $\bhat{f}$ est intégrable sur $P$.
    \end{definition}
    Dans ce cas on pose: 
    \begin{align*} \int_E f\left(\overline{x}\right) d\overline{x} =  \int_P \bhat{f}\left(\overline{x}\right) d\overline{x} \end{align*}
\end{parag}
\begin{parag}{Remarque}
    \begin{enumerate}
        \item La définition ne dépends pas du choix du pavé fermé autour de $E$.
        \item Condition suffisante d'intégrabilité. Si $f : E \to \mathbb{R}$ est bornée sur $E$, continue sur $\overline{E}$et la frontier $\partial E$ est aussi régulière ( de mesure nulle)
    \end{enumerate}
    $\implies $ Alors, $f\left(\overline{x}\right)$ est intégrable sur $E$.
    \begin{subparag}{Frontière régulière}
         \begin{definition}
            \begin{align*} 
                \forall \epsilon > 0\; \; \exists \text{ un recouvrement } \partial E \subset \bigcup_{i \in I}q_i\; , q_i \text{ pavés fermés tels que } \sum_{i \in I}\left|q_i\right| < \epsilon\\
                \implies \partial E \text{ est de mesure nulle }
            \end{align*}
         \end{definition}
    \end{subparag}
    
\end{parag}


