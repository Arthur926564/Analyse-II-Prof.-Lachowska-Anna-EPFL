\lecture{3}{2025-02-24}{EDL1 Et Méthode de démonstration}{}
\subsubsection{Rappel: Equation différentielles linéaires du premier ordre (EDL1)}
\begin{parag}{Rappel}
    \[y' + p(x)y = f(x)\]
    Où $p, f: I \to \R$ fonctions continues. Alors la solution générale est donnée par la formule:
    \begin{formule}
    \[y(x) = y_{hom}(x) + y_{part}(x)\]
    \end{formule}
    Où $y_{hom}(x)$ est la solution générale de l'équation générale de l'équation homogène associée: $y' + p(x)y = 0$ et $y_{part}(x)$ est une solution particulière de l'équation donnée : $y' + p(x)y = f(x)$.
    \begin{itemize}
        \item $y_{hom}(x) = Ce^{-P(x)}$, où $P(x) = \int p(x)dx$ est une primitive (sans constante), $C \in \R$.
        \item $y_{part}(x) = \left(\int f(x) e^{P(x)}dx\right)e^{-P(x)}$
    \end{itemize}
    \begin{theoreme}
        La solution générale de l'EDL1 : 
        \[y(x) = Ce^{-P(x)} + \left(\int f(x)e^{P(x)}dx\right)e^{-P(x)}\]
    \end{theoreme}
    \begin{framedremark}
        Attention avec le signe moins qui se trouve dans la solution homogène mais pas dans la solution particulière.
    \end{framedremark}
    \begin{subparag}{Ex1}
        $y' - \underbrace{\frac{2}{x}}_{p(x)}y = \underbrace{x^3 + 1}_{f(x)}$ avec $p: ]-\infty, o[$ et $]0, \infty[ \to \R$ continue, $f : \R \to R$ est continue.
        \\
        $P(x) = \int -\frac{2}{x} = -2\ln|x| \implies P(x) = -2\ln|x|$ avec $x \neq 0$
        \\
        On a donc comme solution homogène:
        \[y_{hom}(x) = Ce^{-P(x)} = Ce^{--2\ln|x|} = Ce^{--\ln|x|^2} = Ce^{--\ln x^2} = Cx^2\]
        Sur $]-\infty, 0[ \;\cap\; ]0, \infty[$
        \\
        On cherche maintenant une solution particulière de l'équation complète:
        \[y' + \frac{-2}{x}y = x^3 + 1\]
        On utilise la méthode de la variation des constantes:
        \begin{align*}
            \int f(x)e^{P(x)} dx &= \int (x^3 + 1)e^{-\ln x^2}dx \\
            &= \int \frac{x^3 + 1}{x^2} dx \\
            &= \int (x + \frac{1}{x^2})dx \\
            &= \frac{1}{2}x^2 - \frac{1}{x} \; \; \text{pas de constante}
        \end{align*}
        Ce qui implique donc que:
        \begin{align*}
            y_{part}(x) = (\frac{1}{2}x^2 - \frac{1}{x})e^{-(-\ln x^2)} = (\frac{1}{2}x^2 - \frac{1}{x})x^2 = \frac{1}{2}x^4 - x
        \end{align*}
        \textbf{Verification:}
        \\
        \begin{align*}
            y_{part}'(x) - \frac{2}{x}y_{part} &= 2x^3 - 1 - \frac{2}{x}(\frac{1}{2}x^4 - x) \\
            &= 2x^3 - 1 -x^3 + 2 =x^3 + 1
        \end{align*}
        Solution générale de l'équation originale:
        \begin{align*}
            y(x) = Cx^2 + \frac{1}{2}x^4 - x
        \end{align*}
        Sur $]-\infty, 0[$ et sur $]0, \infty[$
    \end{subparag}
    \begin{framedremark}
        Si on mulitplie par $x$ l'équation de base: 
        \[xy' - 2y = x^4 + x\]
        Alors, la solution va sur $\R$
        \\
        $\implies y(x) = Cx^2 + \frac{1}{2}x^4 - x$ sur $\R$
    \end{framedremark}
    \begin{subparag}{Ex2}
        $y' - (\tan x)y = \cos x$ $\tan (x)$ n'est pas continue en $x = (2k + 1)\frac{\pi}{2}, k \in \mathbb{Z}$. Puisque $0 \in ]-\frac{\pi}{2}, \frac{\pi}{2}[ \implies$ on considère l'équation sur $]-\frac{\pi}{2}, \frac{\pi}{2}[$, $p, f: ]-\frac{\pi}{2}, \frac{\pi}{2}[ \to \R$ continues.
        \\
        \begin{enumerate}
            \item Solution générale de l'équation homogène associée.
            \begin{align*}
                y' + (-\tan x)y &= 0 \\
                P(x) = \int (-\tan x)dx &= - \int \frac{\sin x}{\cos x} \\
                &\underbrace{=}_{\int \frac{du}{u}} \int \frac{d(\cos x)}{\cos x} = \ln |\cos x| \\
                &\implies P(x) = \ln (\cos x) \text{ sur } ]-\frac{\pi}{2}, \frac{\pi}{2}[
            \end{align*}
            On a donc:
            \[y_{hom}(x) = Ce^{-P(x)} = Ce^{-\ln \cos x} = \frac{C}{\cos x}, x \in ]-\frac{\pi}{2}, \frac{\pi}{2}[, C \in \R\]
            \\
            \textbf{Vérification:}
            \begin{align*}
                -\frac{C}{\cos^2x}\cdot (-\sin x) - \tan x \cdot \frac{C}{\cos x} = C\frac{\sin x}{\cos^2 x} - C\frac{\sin x}{\cos^2 x} = 0
            \end{align*}
            \item Solution particulière de l'équation complète:
            \[y' - \tan x y = \cos x\]
            Selon la même méthode:
            \begin{align*}
                \int f(x) e^{P(x)}dx = \int \cos x e^{\ln \cos x} dx &= \int \cos^2 x \;dx \\
                &= \int \frac{1}{2}(1 + \cos 2x) dx \\
                &= \frac{1}{2}x + \frac{1}{4}\sin 2x
            \end{align*}
            On a donc:
            \begin{align*}
            y_{part}(x)& = (\frac{1}{2}x + \frac{1}{4}\sin 2x)\cdot e^{-P(x)} \\&=
            (\frac{1}{2}x + \frac{1}{4}\sin 2x)\frac{1}{\cos x}\\
            &= \frac{1}{2}\frac{x}{\cos x} + \frac{1}{4}\frac{2\sin x \cos x}{\cos x} \\
            y_{part}(x) &= \frac{1}{2}\frac{x}{\cos x} + \frac{1}{2}\sin x, \; x \in ]-\frac{\pi}{2}, \frac{\pi}{2}[
            \end{align*}
            
        \end{enumerate}
    \end{subparag}
\end{parag}

\subsection{Application de EDVS (EDL1): Croissance et decroissance exponentielle}
\begin{parag}{Exemple}
    Soit $y = y(t)$ tel que $y' = ky, k \in \R$; $y = 0$ est une solution \\
    EDVS: $\int \frac{dy}{y} = \int k dt \implies \ln |y| = kt + C_1 \implies |y| = e^{C_1}e'{kt} \implies y(t) = Ce^{kt}$
    \\
    Condition initiales : 
    \begin{itemize}
        \item $y(0) = C = y_0 > 0$
        \item $y(t) = y_0e^{kt}$
    \end{itemize}
    La solution maximale satisfaisant la condition initiale $y(0) = y_0$ est:
    \[y(t) = y_0e^{kt}\]
\end{parag}
\subsubsection{Méthodes de démonstration: Disjonction des cas}
\begin{parag}{Méthode 3: Raisonnement par disjonction des cas}
    \begin{definition}
        Soient $P, Q$ deux propositions. Pour montrer que $P \implies Q$ on sépare l'hypothèse de $P$ de départ en différent cas possibles et on montre que l'implication est vraie dans chacun des cas. Il est très important de considérer \important{tous les cas possibles}
    \end{definition}
    \begin{subparag}{Ex1}
        Pour tout $x, y \in \R$ on a:
        \[||x| - |y| | \leq ||x - |\]
        \begin{enumerate}
            \item $|x| \geq |y| \implies$
            \begin{align*}
            ||x| - |y|| &= |x| - |y|\\
            &= |x- y + y| - |y|\\
            &\overbrace{\leq}^{\Delta} |x-y| + |y| - |y| = |x-y|  
            \end{align*}
            \item $|x| < |y| \implies$
            \begin{align*}
                ||x| - |y|| &= -|x| + |y|\\
                &= -|x| + |y -x + x|\\
                &\overbrace{\leq}^{\Delta} -|x| + |y-x| + |x| = |y-x|\\ &= |x-y|
            \end{align*}
        \end{enumerate}
    \end{subparag}
    \begin{subparag}{Ex2}
        Pour tout $n \in \mathbb{Z}$, $2n^2 + n + 1$ n'est pas divisible par $3$. 3 Cas:
        \begin{enumerate}
            \item $n \equiv 0 \mod 3 \iff n = 3k, \; k \in \mathbb{Z}$
            \\
            \begin{align*}
                2n^2 + n + 1 = 2(3k)^2 + (3k) + 1 \equiv 1 \mod 3
            \end{align*}
            \item $n \equiv 1 \mod 3 \iff n = 3k + 1, \; k \in \mathbb{Z}$
            \begin{align*}
                \implies 2n^2 + n + 1 = 2(3k + 1) ^2 + (3k + 1) + 1 \equiv 2 + 1 + 1 \equiv 1 \mod 3
            \end{align*}
            \item $n \equiv 2 \mod 3, n = 3k + 2, k \in \mathbb{Z}$
            \begin{align*}
                2n^2 + n + 1 = 2(3k + 2)^2 + (3k + 2) + 1\equiv  8 + 2 + 1 \equiv 2 \mod3
            \end{align*}
        \end{enumerate}
        Finalement, $2n^2 + n + 1$ n'est pas divisible par $3 \; \forall n  \in \mathbb{Z}$.
    \end{subparag}
\end{parag}
\subsubsection{Méthode de démonstration: équivalence}

\begin{parag}{Méthode 4: Comment démontrer les propositions de la forme $P \iff Q$}
    Deux méthode existent:
    \begin{enumerate}
        \item $P \implies Q$ \important{ET} $Q \implies P$
        \item Suite d'équivalences : $P \iff R_1 \iff R_2 \iff \cdots \iff Q$
    \end{enumerate}
    \begin{framedremark}
        Pour la deuxième méthodes, il faut vérifier que chaque implication est une \textbf{équivalence}.
    \end{framedremark}
    \begin{subparag}{Ex3}
        Soit $a, b \in \mathbb{N}$ : 
        \begin{itemize}
            \item $P : \{ ab + 1 = c^2$ pour un nombre naturel $c \}$
            \item $Q: \{a = b \pm 2\}$
            
        \end{itemize}
        \textbf{Proposition} $P \iff Q$
        \\
        \textbf{Démonstration}
        \begin{align*}
            \underbrace{ab + 1 = c^2}_P \iff ab = c^2 - 1 \iff ab\\
            = (c+1)(c-1) \underbrace{\iff}_{\text{Faux}} \begin{cases}
                \begin{cases}
                    a = c-1 \\
                    b = c + 1 
                \end{cases}\\
                \begin{cases}
                    a = c+1 \\
                    b = c-1
                \end{cases}
            \end{cases}
        \end{align*}
        Néanmoins, Contre exemple:
        $a = 3, b=8$ on a que $24 + 1 = 25  = 5^2 = c^2$, $P$ est vrai, $Q$ est faux\\
        Proposition qui est vraie: $Q \implies P$ Soient $a, b \in \mathbb{N} : a = b \pm 2$, Alors $ab + 1 = c^2, c \in \mathbb{N}$
        \\
        \textbf{Démonstration}
        \begin{align*}
            a = b \pm 2 \implies ab + 1 &= b(b\pm 2) + 1\\ &= b^2 \pm 2b + 1 \\&= (b \pm 1)^2 = c^2
        \end{align*}
    \end{subparag}

    \begin{subparag}{Ex4}
        Soient $z = \rho \underbrace{e^{i\varphi}}_{\rho > 0} \in \mathbb{C}^*$, $P :\{z^2 \in \R^*\}$, $Q: \{\varphi = \frac{\pi k}{2}, k \in \mathbb{Z}\}$
        \\
        On cherche ici à savoir la relation entre $P\; \, ??\;  \,Q$
        \\
        \textbf{Démonstration} $Q \implies P$:
        \\
        Soit $z = \rho e^{i\varphi}, \varphi = \frac{\pi}{2}k \implies z^2 = \rho^2 e^{2i\varphi} = \rho^2(-1)^k \in \R^*$.
        \\
    \textbf{Démonstration} $P \implies Q$
    \\
    Soit $z = \rho e^{i\varphi}, \rho > 0 \implies z^2 = \rho^2e^{2i\varphi}$
    \end{subparag}
    
\end{parag}




