\documentclass[a4paper]{article}

% Expanded on 2025-06-08 at 19:56:46.
\usepackage{graphicx} % Required for inserting images
\usepackage{amsmath}
\usepackage{amsfonts}
\usepackage[bottom=2.5cm, top=2.5cm]{geometry}


\title{Corrigé exa 2021}
\author{Arthur Herbette}
\date{Dimanche 08 juin 2025}

\begin{document}
\maketitle
\paragraph{Question 1}
En premier lieu on voit que dans l'ensemble $A$ on a que $z$ est ``libre''. Dans l'ensemble $B$, $z > 1$ est obligatoire. l'intersection des deux ensemble oblige notre $z > 1$ mais c'est tout. Il peut donc tendre vers l'infini. Dès lors l'intersection n'est pas borné.\\
\paragraph{Question 2}
Ici on utilise lagrange comme d'habitude\\
\begin{align*} 
    \nabla f\left(x, y, z\right) = \left(y, x + 2z, 2y\right)\\
    \nabla g \left(x, y, z\right) = \left(2x, 2y, 2z\right)
\end{align*}
On a donc:
\begin{align*} 
    \begin{cases}
        y =  2\lambda x\\
        x + 2z = 2\lambda y\\
        2y = 2\lambda z
    \end{cases} 
\end{align*}
Si on prends $\lambda \neq 0$ on a donc:
\begin{align*} 
    2\lambda x =  \lambda z \implies x =  \frac{z}{2}
\end{align*}
Ce qui nous amène ensuite aussi à



\end{document}
