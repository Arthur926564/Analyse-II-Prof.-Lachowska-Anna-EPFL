
\chapter{Equations différentielles ordinaires}
\lecture{1}{2025-02-17}{Equa Diff}{}
\section{definition} 
\begin{definition}
    \important{Une équation différentielle ordinaire} est une expression \[E(x, y, y', \dots, y^{(n)}) = 0 \]
    où $E$ est une expression fonctionnelle, $n \in \mathbb{N}_0$, et $y = y (x)$ est une fonction inconnue de $x$ \\
    On cherche un intervalle ouvert $I \subset \mathbb{R}$ et une fonction $y : I \to \mathbb{R}$ de classe $C^n$ telle que l'équation donnée est satisfaite $\forall x \in I$.
\end{definition}
\begin{parag}{Equation à variable séparées} 
 Une équation à variables séparées est une équation du type $f(y)\cdot y' = g(x)$ est une \important{EDVS} où : 
 \begin{itemize}
     \item $f: I \to \mathbb{R}$ est une fonction continue sur $I \subset \mathbb{R}$
     \item $g: J \to \mathbb{R}$ est une fonction continue sur $J \subset \mathbb{R}$
 \end{itemize}
 Une fonction $y: J' \subset J \to \mathbb{R}$ de classe $C'$ satisfaisant l'équation $f(y)\cdot y' = g(x)$ est une solution
 \begin{subparag}{Remarque personnelle}
     \begin{framedremark}
         Ce type d'équation se résoudre très rapidement car on peut transformer le $y'$ en $\frac{dy}{dx}$ et "mettre le $dx$ de l'autre côté" : 
         \[f(y)\cdot\frac{dy}{dx} = g(x) \implies \int f(y)dy = \int g(x) dx\] 
         Et il suffit donc t'intégrer les deux côtés et le tour est joué.
     \end{framedremark}
 \end{subparag}
\end{parag}

\begin{parag}{Terminologie}
    Soit $E(x, y, \dots, y^{(n)}) = 0$ \textcolor{red}{(*)} une équation différentielle (ED):
    \begin{itemize}
        \item \textbf{Def:} un nombre naturel $n \in \mathbb{N}_+$ est \important{l'ordre} de l'équation $(*)$ si $n$ est l'ordre maximal de dérivée de $y(x)$ dans l'équation.
        \item \textbf{Def:} Si (*) est de la forme $\alpha_0(x)y + \alpha_1(x)y' + \alpha_2(x)y'' + \cdots + \alpha_n(x)y^{(n)} = b(x)$ alors l'équation est dire \important{linéaire} où $\alpha_i(x)$, $b(x)$ dont des fonctions continues
        \item \textbf{Def} Si l'expression (*) ne contient pas de $x$ l'équation (*) est dire \important{autonome}
    \end{itemize}
\end{parag}
\begin{parag}{Problème de Cauchy}
    \begin{definition}
        Résoude \important{Le problème de Vauchy (ED avec des conditions initiales)} pour l'équation $E(x, y, y', \dots, y^{(n)}) = 0$ c'est de trouver l'intervalle ouvert $I \subset \mathbb{R}$ et une fonction $y : I \to \mathbb{R}$ de classe $C^n (I)$, telle que $E(x, y, \dots, y^{(n)}) = 0$ sur $I$ et $y(x_0) = b_0$, $y_(x) = B, \dots, y'(x_2) = \dots$
    \end{definition}
    Le nombre des conditions initiales depend du type de l'ED
    \begin{framedremark}
        C'est ce qui se passe en physique lorsqu'on a une forme et que l'on chercher la position au court du temps:
        \begin{align*}
            ma &= F\\
            a &= \frac{F}{m}\\
            \frac{d^2x}{dt^2} &= \frac{F}{m}\\
            x = \frac{1}{2}\frac{F}{m}t^2 + c_1 t + c_0
        \end{align*}
        Et le but est de trouver ses constantes qui sont les conditions initiales.
    \end{framedremark}
    \begin{definition}
        Une solution d'un problème de Cauchy est \important{maximale} si elle est définie sur le plus grand intervalle possible.
    \end{definition}
\end{parag}
\section{Existence et unicit d'une solution de EDVS}
\begin{parag}{Théorème}
    \begin{theoreme}
        Soit \begin{itemize}
            \item $f: I \to \mathbb{R}$ une fonction continue telle que $f(y) \neq 0\;\; \forall y \in I$
            \item $g : J \to \mathbb{R}$ une fonction continue. Alors pour tout couple $(x_0 \in J, b_0 \in I)$, l'équation $f(y)\cdot y' = g(x) (**)$ admet une solution $y : J' \subset J \to I$ vérifiant la condition initiale $y(x_0) = b_0$
        \end{itemize}
        Si $y_1 : J_1 \to I$ et $y_2 : J_2 \to I$ sont deux solutions telles que $y_1(x_0) = y_2(x_0) = b_0$, alors $y_1(x) = y_2(x)$ pour tout $x \in J_1 \cap J_2$
        \\
        (Demonstration la prochaine fois
        
    \end{theoreme}
\end{parag}
\begin{parag}{Introduction}
    \begin{definition}
        Une \important{proposition} est un énoncé qui peut être vrai ou faux.
    \end{definition}
    \begin{definition}
        Une \important{démonstration} est une suite d'implication logique qui sert à dériver la proposition en question à partir des axiomes (propositions admises comme vraies) et des propositions préalablement obtenue
    \end{definition}
\end{parag}
\section{Méthode de démonstration}
\begin{parag}{Méthode 1}
    Démonstration direct:\\ $\underbrace{P}_{\text{condition donnée}}$ $\implies$ implications logiques/axiomes/propositions connues $\implies \underbrace{Q}_{\text{proposition désirée}}$
    \begin{subparag}{Remarque personnelle}
        C'est pas vraiment très claire comme ça mais en gros ça veut juste dire que pour prouver quelque chose on y va en mode brute force (tout les nombres entiers sont des nombres réels (propositions connues) et par exemple est ce que $23$ est un réel?) 
    \end{subparag}
\end{parag}
\begin{parag}{Raisonnement par contraposée}
    Comme vu en AICC on sait que $P \implies Q \equiv \neg P \implies \neg Q$
\end{parag}
