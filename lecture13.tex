\lecture{13}{2025-04-07}{Jacobienne}{}
\subsection{Fonction à valeur dans $ \mathbb{R}^m$, $m \geq 1$, la matrice jacobienne}
\begin{definition}
    Plus généralement, on peut considérer les fonctions:
    \begin{align*}
        \overline{f} \cdot E^{ \subset \mathbb{R}^m}
    \end{align*}
    
    \begin{align*}
        \overline{f}( \overline{x}) = \begin{pmatrix}
            f_1( \overline{x}) \\
            f_2( \overline{x})\\
            \vdots \\
            f_m( \overline{x})
        \end{pmatrix}
    \end{align*}
    Chaque fonction $f_i$ est une fonction réelle de $n$ variables réelles.
\end{definition}

\begin{parag}{Dérivée}
    La $k$-ième dérivée partielle de $f: E \to \mathbb{R}^m$ en $ \overline{a} \in E$ est:
    \begin{align*}
        \frac{\partial \overline{f}}{\partial x}( \overline{a}) = \begin{pmatrix} \frac{\partial f}{\partial x_k}( \overline{a})       \\
            \frac{\partial f}{\partial x_k}( \overline{a})\\
            \vdots\\
            \frac{\partial f}{\partial x_k}( \overline{a})
        \end{pmatrix}
    \end{align*}
    

\end{parag}

