\lecture{19}{2025-04-28}{Test blanc}{}
\subsection{Corrigé du test blanc}
Ici ce n'est pas le corrigé de la professeur mais moi directement lorsque je fais mes exercices donc il se peut que l'approche ne soit pas rigoureuse peut être même éronnée en quelque sorte:
\begin{parag}{Question 1}
   La solution $y\left(x\right)$ de l'équation différentielle:
   \begin{align*} y'\left(x\right) = \frac{x}{x^2 + 9}\left(y\left(x\right) - 1\right) \end{align*}
   qui satisfait la condition initiale $y\left(0\right) = 7$ vérifie aussi pour $y\left(4\right) = \ldots$.\\
   \begin{subparag}{Corrigé}
       Donc on voit ici en premier lieu que l'équation est une EDVS, on arrive à séparer les $y$ des $x$:
       \begin{align*} y'\left(x\right) = \frac{x}{x^2 + 9}\left(y - 1\right) \\
       \frac{y'}{y - 1} = \frac{x}{x^2 + 9}
    \end{align*}
       A partir de là, il faut alors intégrer des deux côtés et on trouvera la solution:
       \begin{align*} 
            \int \frac{1}{y - 1} dy = \int \frac{x}{x^2 + 0} dx\\
           \ln \left|y - 1\right| + C' = \frac{1}{2}\ln \overbrace{\left|x^2 + 9\right|}^{\geq 0} + C\\
       \left|y - 1 \right| =  \ln C\overbrace{\sqrt{x^2 + 9}}^{\geq 0} \\
       y - 1 = C\sqrt{x^2 + 9}\\
       y = 1 + C \sqrt{x^2 + 9}
   \end{align*}
   Ensuite on va donc trouver la valeur pour $y\left(0\right)$:\\
   On évalue $y\left(0\right) = 1 + C \cdot \sqrt{9} = 1 + 3C \implies C = 2$. Et donc On évalue notre fonction en $y\left(4\right) = 1 + 2\sqrt{16 + 9} = 11$
   \end{subparag}

    
\end{parag}

\begin{parag}{Question 2}
    La solution $y\left(x\right)$ de l'équation différentielle:
    \begin{align*} y' + \frac{1}{x}y = \frac{2}{\left(1 + x^2\right)^2} \end{align*}
    sur l'intervalle $] 0, + \infty [$ qui satisfait la condition initiale $y\left(1\right) = 0$...\\
    Pour cette question soit on apprends par coeur la formule soit on la construit.\\
    On a une équation du type $y' + p\left(x\right)y = f\left(x\right)$. On sait que la solution générale est donné par:
    \begin{align*} y\left(x\right) = y_{hom}\left(x\right) + y_{part}\left(x\right) \end{align*}
    Commençons par la solution homogène:
    \begin{align*} y' + \frac{y}{x} = 0 \end{align*}
    On arrive assez aisément à la séparer:
    \begin{align*} y' &= -\frac{y}{x}\\
    \frac{y'}{y} &= -\frac{1}{x}\end{align*}
    Ensuite on intègre des deux côtés:
    \begin{align*} \int \frac{1}{y} dy = - \int \frac{1}{x}\\
        \ln \left|y\right| = - \ln \left|x\right| + C''
    \end{align*}
    Et à partir d'ici (je vais le faire très proprement un peu overkill)
    \begin{align*} e^{\ln \left|y\right|} = e^{ \ln \left|\frac{1}{x}\right| + C''}\\
        y = e^{C''}e^{\ln \left|\frac{1}{x}\right|}\\
        y = C'''\frac{1}{x}
    \end{align*}
    Après avoir trouver la solution homogène associé $y_{hom} = C\frac{1}{x}$, On utilise la méthode de la variation de la constante pour trouver la solution particulière:
\begin{align*} y_{part}\left(x\right) = c\left(x\right)y_{hom}\left(x\right) \end{align*}
Et par le cours (section \ref{subsec:variationconstante}) on cherchera donc la fonction $c\left(x\right)$:
\begin{align*}c\left(x\right) = \int f\left(x\right) e^{P\left(x\right)}  \end{align*}
On voit ici qu'il faudra faire attention au signe qu'on avait enlevé dans le $\ln$.
Donc on avait trouver pour $P\left(x\right) = \ln \left|Cx\right| $ (on peut mettre le C dedans si on fait une manip avec l'exponentielle).
        \begin{align*} c\left(x\right) &= \int \frac{2x}{\left(1 + x^2\right)^2} dx
        \end{align*}
        Ici j'ai essayé avec les éléments simples mais j'ai rien trouvé donc je vais essayé avec un changement de variable:\\
        Soit $u = 1 + x^2$, on prends $\frac{du}{dx} = 2x$ si on remplace:
            \begin{align*} c\left(x\right) &= \int \frac{2x}{u^2}\frac{du}{2x}\\
            &= \int \frac{1}{u^2}du\\
            &= -\frac{1}{u} + C = -\frac{1}{\left(1 + x^2\right)}
        \end{align*}
        On obtient donc pour la solution particulière:

        \begin{align*} y_{part}\left(x\right) = -\frac{1}{1+x^2}\cdot \frac{1}{x} \end{align*}
On a donc la solution générale:
\begin{align*} y\left(x\right) = \frac{1}{x\left(1+x^2\right)} + \frac{C}{x} \end{align*}
avec $C = \frac{1}{2}$ 

\end{parag}
\begin{parag}{Question 3}
    La solution de $y\left(x\right)$:
    \begin{align*} y'' - y' - 6y = 4e^{-x}
    \end{align*}
    On trouve d'abord la solution homogène:
    \begin{align*} \lambda^2 - \lambda - 6 = 0 \end{align*}
    Qui a pour racine $3$ et $-2$:
    \begin{align*} y_{hom} = C_1e^{3x} + C_2e^{-2x} \end{align*}
    Pour la solution particulière, on pose $y = Ae^{-x}$ avec:\\
    \begin{align*} y' = -Ae^{-x}\;\;\; y'' = Ae^{-x} \end{align*} 
    Si on rempli l'équation on obtient:
    \begin{align*} Ae^{-x} + Ae^{-x} -6Ae^{-x} = 4e^{-x}\\
    2A - 6A = 4\\
    A = -1
\end{align*}
Et donc on trouve que:
\begin{align*} y\left(x\right) = -e^{-x} + C_1e^{3x} + C_2e^{-2x} \end{align*}
Et donc on posera pour les deux conditions tel que:
\begin{align*} 
    \begin{cases}
        -2 = -1 + C_1 + C_2\\
        3 = 1 + 3C_1 - 2C_2
    \end{cases}
   \\
   \begin{cases}
       C_1 = 0\\
       C_2 = -1
   \end{cases}
\end{align*}
Ce qui donne comme sollution final:
\begin{align*} y\left(x\right) = -e^{-x} - e^{-2x} \end{align*}
\end{parag}



\begin{parag}{Question 4}
    Le sous ensemble:
    \begin{align*} E = \{ \left(x, y\right) \in \mathbb{R}^2: x > 1 \text{ et } -1 < xy < 1\} \subset \mathbb{R}^2 \end{align*}
    Est 
    \begin{itemize}
        \item ouvert et borné
        \item fermé et non borné
        \item fermé et borné
        \item ouvert et non borné
    \end{itemize}
    Ici déja on peut voir en quelque sorte a l'oeil nu si le sous-ensemble est ouvert ou fermé. Si l'on le regard comme un ensemble dans $\mathbb{R}$ on voit que tout les $<$ ou $>$ donne des ensemble ouvert (de ce type ] [) ce qui est ``l'équivalent'' de ouvert. et donc peut deviner ici que l'ensemble est ouvert et non fermé.\\
    Pour ce qui est de borné, prenons la deuxième condition, pour montrer qu'un ensemble n'est pas borné il faut montrer qu'il y a une de ces variables qui peut s'étendre jusqu'à l'infini (c'est une sorte d'esquisse de preuve mais ca permet de se répérer lors de qcm).\\
    Donc on prend la deuxième condition: $-1 < xy < 1$ si on prends par exemple: $y = \frac{1}{k + 1}$ et $x = k$ et que nous faisons la limite du produit:
    \begin{align*} \lim_{k \to \infty} \frac{k}{k+1} \end{align*}
    On voit que cette limite est bornée par $1$: soit $k \in \mathbb{R}$prenons:
    \begin{align*} 0 < 1\\
k < k + 1\\
\frac{k}{k+1} < 1
    \end{align*}

    On voit ici donc que $x$ peut tendre vers l'infini et fera toujours parti de l'ensemble, donc l'ensemble n'est pas bornée.

    
    
\end{parag}
\begin{parag}{Question 5}
    Soit $\{\overline{x}_n\}$ la suite d'éléments de $\mathbb{R}^2$ définie par
    \begin{align*} \overline{x_n} = \left(n\sin\left(\frac{\left(-1\right)^n}{n}\right), \frac{\left(-1\right)^n}{n}\sin\left(n\right)\right)^T \end{align*}
    Et on demande la convergence de la suite et/ou elle est bornée.\\
    Pour rappel lorsqu'on traite de suite dans $\mathbb{R}^n$ c'est comme si on traite de suite individuelle donc prenons la première:
    On reconnaît ici bien le schéma de la suite:
    \begin{align*} \lim_{x \to 0} \frac{\sin\left(x\right)}{x} = 1 \end{align*}
    Si on prends la premiere suite on voit que cela ressemble dangereusement comme ceci:
    \begin{align*} \frac{\sin\left(\frac{\left(-1\right)^n}{n}\right)}{\frac{1}{n}} \end{align*}
    Néanmoins à cause du $\left(-1\right)^n$ la suite converge \important{absolument} vers 1 mais oscille entre -1 et 1, donc la suite est bornée mais pas convergente.\\
    Pour l'autre On peut aussi utilise le même arguement mais dans ``l'autre sens'' car on a $\frac{\left(-1\right)^n}{n}$ qui tends vers 0 mais $\sin\left(n\right)$ quant a lui va juste oscillé à l'infini, donc la suite ne pourra jamais converger vers une valeur.\\
    \begin{align*} \left|\frac{1}{n}\sin\left(n\right)\right| \leq \frac{n}{n} \end{align*}
    Et on arrive pas a montrer que la limite tends vers 0 (le sinus se ``comporte'' comme la variable a l'intérieur lorsque celle ci s'approche de 0).
    Et donc la suite est bornée mais pas convergente


\end{parag}


\begin{parag}{Question 6}
    Soit la fonction $f: \mathbb{R}^2 \to \mathbb{R}$ définie par:
    \begin{align*} 
\begin{cases}
    \frac{x^3y + y^3x}{x^2 + y^2} \; \; \text{ si } \left(x, y\right) \neq \left(0, 0\right)\\
    0  \; \; \text{ si } \left(x, y\right) = \left(0, 0\right)
\end{cases}
    \end{align*}
    Comme je soupçonne grandement que la fonction soit dérivable je vais commencer par ça:\\
 Soit $r\left(\overline{x}\right) = f\left(\overline{x}\right) - f\left(\overline{a}\right) - < \nabla f\left(\overline{a}\right), \overline{x}- \overline{a}>$.\\
 Pour ce qui est du gradient, on peut le faire de tête car il y a une puissance de 3 en haut et une puissance de 2 en bas. donc quand tu feras la limite tu tomberas sur 0 des deux côtés. donc $r\left(\overline{x}\right) =  f\left(\overline{x}\right)$
    
 \begin{align*} \lim_{\left(x, y\right) \to \left(0, 0\right)} f\left(x, y\right) = \lim_{r \to 0} \frac{r^4\cos^3\psi\sin\psi + r^4\sin^3\psi\cos\psi}{r^2}\\
     = \lim_{r \to 0} r^2\left(blabla\right) = 0
\end{align*}
Donc la fonction est dérivable.
\end{parag}
\begin{parag}{Question 7}
    Soit la fonction
    \begin{align*} f\left(x, y\right) = \sin\left(\pi x^2 y\right) \end{align*}
    On demande la dérivée directionnelle $D_{\overline{v}}f\left(1, 1\right)$ en suivent le vecteur $\overline{v} = \left(-\frac{1}{\sqrt{5}}, \frac{2}{\sqrt{5}}\right)^T$\\
    Par la definition:
    \begin{align*} \lim_{h \to 0}\frac{ f\left(\overline{e} + h \overline{v}\right)}{h} \end{align*}
    Ce qui nous donne:
    \begin{align*} \lim_{h \to 0} \frac{\sin\left(\pi\left(1-\frac{h}{\sqrt{5}}\right)^2\left(1 + \frac{2h}{\sqrt{5}}\right)\right)}{h}  \end{align*}

    Ce qui après des simplifications:
    \begin{align*}\lim_{h \to 0} \frac{\sin\left(\pi + \frac{-3h^2}{5} + \frac{2h^3}{5\sqrt{5}}\right)}{h}  \end{align*}
    qui nous donne avec les identités trigonométrique:
    \begin{align*} \lim_{t \to 0} \frac{\cos \pi \sin\left(-\frac{3}{h^2} + \frac{2h^3}{5\sqrt{5}}}{h} = 0 \end{align*}
On sait que c'est $0$ parce qu'il y a une puissance $3$ dans le sinus qui va dépasser la puissance $1$ qui est en bas.
\end{parag}
\begin{parag}{Question 8}
    Donc ici il suffit de trouver la jacobienne de $\overline{f}\left(x, y\right)$:
    On calcule donc directement:
    \begin{align*} \nabla f_1\left(x, y\right) = \left(x, y\right)\\
    \nabla f_2\left(x, y\right) = \left(2x, 2y\right)\end{align*}
    Ce qui donne lorsqu'on met en matrice et qu'on évalue en $\left(1, 1\right)$:
    \begin{align*} J_{\overline{f}\left(1, 1\right)} = \begin{pmatrix} 1 & 1 \\ 2 & 2 \end{pmatrix}  \end{align*}
    Et comme on demande directement le gradient de la fonction composée de matrice qui font $1 \times 2 $ avec $2 \times 2$ on a juste besoin de faire le produit matriciel de nos deux jacobienne:
    \begin{align*} \nabla h\left(1, 1\right) = \begin{pmatrix} 1 & -\frac{1}{4} \end{pmatrix} \times\begin{pmatrix} 1 & 1 \\ 2 & 2 \end{pmatrix} = \begin{pmatrix} \frac{1}{2} & \frac{1}{2} \end{pmatrix}  \end{align*}
    Je note juste ici qu'on a pas eu besoin de faire d'autre calcul car aussi $f\left(1, 1\right) = \left(1, 2\right)$ qui était exactement le point du gradient de $f\left(x, y\right)$.
\end{parag}




