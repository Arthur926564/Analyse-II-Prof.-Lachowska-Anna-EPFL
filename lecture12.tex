\lecture{12}{2025-03-26}{Tangente de la surface}{}
\begin{parag}{Rappel}
    \begin{definition}
        Soit $f: E \to \mathbb{R} , \; E \subset \mathbb{R}^n $ ouvert, $a \in E$.\\
        On dit que $f$ est dérivable au point $ \overline{a}$ s'il existe une transformation linéaire $L_{ \overline{a}}: \mathbb{R}^n  \to \mathbb{R}$ et une fonction $r: E \to \mathbb{R}$ telles que:
        \begin{align*}
            f( \overline{x}) = f( \overline{a}) + L_{ \overline{a}}( \overline{x} - \overline{a}) + r( \overline{x}) \; \forall \overline{x} \in E\\
            \lim_{ \overline{x} \to \overline{a}} = \frac{r( \overline{x})}{ \mid \mid \overline{x} - \overline{a}\mid \mid} = 0
        \end{align*}
        
    \end{definition}
    

\end{parag}

\begin{parag}{Théorème $1$}

    \begin{theoreme}
        Soit $f: E \to \mathbb{R}$, $ \overline{a} \in E$ tel que $f$ est dérivable en $ \overline{a}$ de différentielle $L_{ \overline{a}}: \mathbb{R}^n \to \mathbb{R}$. Alors:
        \begin{itemize}
            \item $f$ est continue en  $\overline{a} \in E$
            \item Pour tout $\overline{v} \in \mathbb{R}^n , \overline{v} \neq \overline{0}$, la dérivée directionnelle $Df( \overline{a}, \overline{v})$ existe et 
                \begin{align*}
                    Df( \overline{a}, \overline{v}) = L_{ \overline{a}}( \overline{v})
               \end{align*}
           \item Toutes les dérivées partielles existent de $f$ en $\overline{a}$ et
               \begin{align*}
                   \frac{\partial f}{\partial x_k}( \overline{a}) = L_{ \overline{a}}( \overline{e}_k)
               \end{align*}
               Le gradient de $f$ existent en $ \overline{a}$ et:
               \begin{align*}
                   \nabla f( \overline{a}) = \left( L_{ \overline{a}}( \overline{e_1}),  L_{ \overline{a}}( \overline{e_2}), \dots,  L_{ \overline{a}}( \overline{e_n}) \right)
               \end{align*}
           \item Pour tout $\overline{v} \in \mathbb{R}^n $, $ \overline{v} \neq \overline{0}$
               \begin{align*}
                   Df( \overline{a}, \overline{v}) = L_{ \overline{a}}( \overline{v}) = < \nabla f( \overline{a}), \overline{v}>
               \end{align*}
           \item Pour tout $ \overline{v} \in \mathbb{R}^n $, $ \mid \mid \overline{v} \mid \mid = 1$, on a que:
               \begin{align*}
                   Df( \overline{a}, \overline{v}) \leq \mid \mid \nabla f( \overline{a}) \mid \mid
               \end{align*}
               Et que si:
               \begin{align*}
                   Df \left( \overline{a}, \frac{\nabla f( \overline{a})}{ \mid \mid \nabla f( \overline{a}) \mid \mid} \right)= \mid \mid \nabla f( \overline{a}) \mid \mid
               \end{align*}
              Alors le gradient donne la direction de la plus grande croissance de $f$ en $\overline{a}$ 
        \end{itemize}

    \end{theoreme}
    
\end{parag}

\subsubsection{Equation du plan tangent de la surface}
\begin{definition}
    Soit $\overline{a}$: $F( \overline{a}) = 0$, $F: \mathbb{R}^n  \to \mathbb{R}$ dérivable en $\overline{x} = \overline{a}$ et $ \nabla F( \overline{a}) \neq 0$:\\
   L'équation de l'hyperplan tangent à $F( \overline{x}) = 0$ au point $\overline{a}$ est:
   \begin{align*}
       < \nabla F( \overline{a}),  ( \overline{x} - \overline{a})> =  0
   \end{align*}
   Et si $F(a, b, c) = 0$ et $ \nabla F(a, b, c) \neq 0$ Alors:
   \begin{align*}
       < \nabla F(a, b, c), (x-a, y-b, z-c) >\; = 0
   \end{align*}
   
\end{definition}
\begin{framedremark}
    Ce qu'on fait en \textit{gros}" c'est de prendre le gradient qui donne le vecteur normal au plan tangent qui a donc dans ces coordonnées les valeurs pour l'équation cartésienne du plan, et on fait comme un changement de référentiel pour pouvoir avoir le $0$ du plan à l'endroit ou il touche le point, c'est de là que provient le $x - x_0, y - y_0, z - z_0$.
\end{framedremark}





\begin{parag}{Résumé}
\begin{subparag}{Dérivée partielles}
   \begin{align*}
       \frac{\partial f}{\partial x_k}( \overline{a}) = \lim_{t \to 0} \frac{f( \overline{a} + t \overline{e_k})- f( \overline{a}) }{t}
   \end{align*}
   si la limite existe, $ \overline{e_k} = (0, \dots, \overbrace{1}^{k}, \dots, 0)$.\\
   Le gradient:
   \begin{align*}
       \nabla f( \overline{a}) = \left( \frac{\partial f}{\partial x_1}( \overline{a}), \frac{\partial f}{\partial x_2}( \overline{a}), \dots, \frac{\partial f}{\partial x_n}( \overline{a}) \right)
   \end{align*}
\end{subparag}

\begin{subparag}{Dérivée directionnelles}
    \begin{itemize}
        \item \begin{align*}
        Df( \overline{a}, \overline{v}) = \lim_{t \to 0} \frac{f( \overline{a} + t \overline{v}) - f( \overline{a})}{t}
    \end{align*}
    Si la limite existe, $ \overline{v} \in \mathbb{R}^n , \overline{v} \neq \overline{0}$.
\item $Df( \overline{a}, \overline{e}_k) = \frac{\partial f}{\partial x_k}( \overline{a})$ si $Df( \overline{a}, \overline{v})$ existent pour tout $ \overline{v} \in \mathbb{R}^n $
    \item Si $f$ est dérivable en $\overline{a}$, alors par le théorème $1$, $f$ est continue en $\overline{a}$, $Df( \overline{a}, \overline{v})$ existe, et on a:
\begin{align*}
    L_{ \overline{a}} ( \overline{v}) = Df( \overline{a}, \overline{v}) = < \nabla f( \overline{a}), \overline{v}>
\end{align*}

    
    
    \end{itemize}
\end{subparag}
\end{parag}

\begin{parag}{Théorème deux}
    \begin{theoreme}
        Soit $E \subset \mathbb{R}^n $ ouvert,  $f: E \to \mathbb{R}$, $\overline{a} \in E$. Supposons qu'il existe $ \delta > 0$ tel que toutes les dérivées partielles $ \frac{\partial f}{\partial x_k}( \overline{x})$ existent sur $B( \overline{a}, \delta)$ et sont continues en $\overline{a}$.\\
        Alors $f$ est dérivable en $\overline{a} \in E$
    \end{theoreme}
    \begin{subparag}{Note personnelle}
        Cela est vrai car, une fonction est dite de classe $C^1$ si au point donnée, toutes les dérivées partielles existent et sont continues. Dès lors, dans notre cas nous avons que notre fonction est de classe $C^1$ au alentour de notre point $\overline{a}$ dès lors elle est aussi dérivable.
        \begin{align*} C^1\left(\overline{a}\right)  \implies \text{ dérivable en } \overline{a} \end{align*}
    \end{subparag}
    
\end{parag}

