\lecture{2}{2025-02-19}{EDO}{}
\subsection{Théorème Existence et unicité d'une solution de EDVS}
\begin{parag}{Théorème}
    \begin{theoreme}
        Soit $f: I \to \mathbb{R}$ une fonction continue telle que $f(y) \neq 0 \; \; \forall y \in I$
        \\
        $g : J \to \mathbb{R}$ une fonction continue. Alors pour tout coupe $(x_0 \in J, b_0 \in I)$, l'équation
        \[f(y)\cdot y'(x) = g(x)\]
        admet une solution $y : J' \subset J \to I$ vérifiant la condition initiale $y(x_0) = b_0$.
        \\
        Si $y_1: J_1 \to I $ et $y_2 : J_2 \to I$ sont deux solutions telles que $y_1(x_0) = y_2(x_0) = b_0$, alors $y_1 (x) = y_2(x)$ pour tout $x \in J_1 \cap J_2$
    \end{theoreme}
\end{parag}
\begin{parag}{Démonstration}
    \begin{framedremark}
        Idée : $\int f(y)dy = \int g(x)dx \implies F(y) = G(x) \implies y(x) = F^{-1}(G(x))$
    \end{framedremark}
    Le reste de la preuve se trouve sur les pdf de Joachim Favre.
\end{parag}
\begin{parag}{Résumé}
\begin{resume}
        EDVS:
        $f(y)\cdot y' = g(x)$ où $f: I \to \mathbb{R}$ continue (respectivement $J$ pour $g$), 
        \\
        Pour résoudre $\int f(y)dy = \int g(x)dx$
        où $\int f(y)dy$ est une primitive (sans constante) et $\int g(x)dx$ est une primitive générale (avec une constante)
    \end{resume}    
\end{parag}
\begin{parag}{Exemple}
    \begin{subparag}{Exemple 1}
        $\frac{y'(x)}{y^2(x)} = 1$ EDVS : $\frac{1}{x^2}$ est contiue sur $\mathbb{R}_+^*$ et $\R_-^*$
        \\
        On a aussi que $g(x)$ est continue sur $\R$. on fait donc:
        \begin{align*}
            \int \frac{1}{y^2}dy = \int dx \implies -\frac{1}{y} = x + C \\
            y = -\frac{1}{x+C} \; \; \forall C \in \R
        \end{align*}
        la solution générale sur $]-\infty, -C [ $ et $] -C, \infty [$.
        \\
        Condition initiale $y(0) = b_0 \in \R^* \implies y(0) = -\frac{1}{C} = b_0 \implies C = -\frac{1}{b_0}$
        \\
        \begin{itemize}
            \item Si $b_0 > 0 \implies \frac{1}{b_0} > 0  \implies y(x) = -\frac{1}{x - \frac{1}{b_0}}$ sur $] -\infty, \frac{1}{b_0} [$ - la solution particulière
            \item Et vis versa pour $b_0 < 0$
        \end{itemize}
    \end{subparag}
\end{parag}
\subsection{Solution maximale}
\begin{parag}{Solution maximale}
    \begin{definition}
        Une solution \important{solution maximale} de l'EDVS avec la condition initiale $y(x_0) = b_0$, $x \in J, b_0 \in I$ est une fonction $y(x)$ de classe $C^1$ satisfaisant l'équation, la condition initiale et qui est définie sur le plus \textbf{grand} intervalle possible. \\
        Le théorème sur EDVS dit que si $f(y) \neq 0$ sur $I$, alors il existe une unique solution maximale. Toute solution avec la même condition initiale est une restriction de la solution maximale
    \end{definition}
\end{parag}
\begin{parag}{Exemple 2}
    L'équation différentielle $2yy' = 4x^3$ avec la condition initiale $y(0) = 0$ possède : 
    \begin{enumerate}
        \item Une seul solution sur $\R$
        \item $2$ solutions sur $\R$
        \item $3$ solutions sur $\R$
        \item $4$ solutions sur $\R$
    \end{enumerate}
    En premier lieu il faudra résoudre:
    \begin{align*}
        \int 2ydy &= \int 4x^3 dx \\
        y^2 &= x^4 + C \; \; \forall C \in \R \\
        y &= \pm \sqrt{x^4 + C} \\
        y(0) &= \pm \sqrt{C'} = 0 \implies C' = 0 \\
        y(x) &= \pm \sqrt{x^4} = \pm x^2
    \end{align*}
    On voit ici qu'il y a $4$ solutions à cause des $\pm$ qui se rajoute entre eux:
    \begin{itemize}
        \item $y(x) = x^2, x \in \R$
        \item $y(x) = -x^2, x \in \R$
        \item $y(x) = \begin{cases} -x^2, x \leq 0 \\ x^2, x > 0\end{cases}$
        \item $y(x) = \begin{cases}
            x^2, x \leq 0 \\ -x^2 , x > 0
        \end{cases}$
    \end{itemize}
\end{parag}

\section{Equation différentielle linéaire du premier ordre (EDL1)}
\begin{parag}{Definition}
    \begin{definition}
    Soit $I \subset \R$ un intervalle ouvert. Une équation de la forme:
    \[y'(x) + p(x)y(x) = f(x), \text{ où } p, f: I \to \R \text{ sont continues }\]
    est une \important{équation différentielle linéaire du premier ordre (EDL1)}
    \end{definition}
    Une solution est une fonction $y: I \to \R$ de classe $C^1$ satisfaisant l'équation.
\end{parag}
\begin{parag}{Comment résoudre une EDL1}
    Considérant l'équation $y'(x) + p(x)y(x) = 0$
    \\
    Elle s'appelle \important{l'équation homogène associée} à l'EDL1 $y' + py = f$ qui nous amène : 
    \[\begin{cases}
        y (x) = 0 \; \; \forall x \in I \\
        \frac{y'(x)}{y(x)} = -p(x) \; \; \; EDVS \implies \int \frac{dy}{y} = -\int p(x) dx
    \end{cases}\]
    Ce qui implique que $\ln |y| = -P(x) + C_1$ où $P(x)$ est une primitive de $p(x)$, $C_1 \in \R$, ensuite, $|y| = e^{-P(x) + C_1} = e^{C_1}e^{-P(x)} \implies y(x) = \pm C_2 e^{-P(x)}, C_2 \in \R_+^*$
    \\
    Mais on a aussi $y(x) = 0$ sur $I$ ce qui implique que 
    \[y(x) = Ce^{-P(x)}\]
    où $C \in \R$, $x \in I$ est la solution générale de l'équation homogène associée $y' + py = 0$ sur $I$
\end{parag}
\subsection{Principe de superposition de solutions}
\label{subsec:variationconstante}

\begin{parag}{Principe}
    Soit $I \subset \R$ ouvert, $p, f_1, f_2 : I \to \R$ fonctions continues
    \\
    Supposons que $v_1: I \to \R$ de classe $C^'$ est une solution 
    \[y' + p(x)y(x) = 0\]
\end{parag}
\begin{parag}{Méthode de la variation de constante}
    On cherche une solution particulière de $y'(x) + p(x) y(x) = f(x) : p, f :I\underbrace{ \to \R}_{\text{continue}}$ sout la forme : 
    \\
    \textbf{Ansatz:}
    \[v(x) = \textcolor{red}{C(x)}e^{-P(x)}\] 
    où $P(x)$ est une primitive de $p(x)$ sur $I$
    \\
    Si $v(x)$ est une solution $\implies v'(x) + p(x)v(x) = f(x)$
    ce qui implique que 
    \[C'e^{-P(x)} + C(x)(-e^{-P(x)})\cdot p(x) + p(x)Ce^{-P(x)} = f(x)\]
    \\
    Ce qui revient a dire 
    \[C'(x) = f(x)e^{P(x)} \implies c(x) = \int f(x)e^{P(x)} dx\]
    une solution particulière de l'équation $y'(x) + p(x) y(x) = f(x)$ est $v(x) = \left(\int f(x) e^{P(x)}dx\right)\cdot e^{-P(x)}$ où $P(x)$ est une primitive de $p(x)$ sur $I$
\end{parag}
\subsection{Théorème à savoir pour l'examen}
\begin{parag}{Proposition}
    Soit $p_1, f : I \to \R$ fonctions continues. Supposons que $v_0 : I \to \R$ est une solution partiulière de l'équation $y'(x) + p(x)y(x) = f(x)$
    \\
    Alors la solution générale de cette équation est:
    \[v(x) = v_0(x) + Ce^{-P(x)}, \text{ pour tout } C \in \R, \text{ où } P(x) \text{ est une primitive de p(x) sur } I\]
\end{parag}
\begin{parag}{Démonstration}
    \textbf{(1)} \\
    Soit $v_1(x)$ une solution de $y'(x) + p(x)y(x) = f(x)$. On va démontrer qu'il existe $C \in \R$ tel que $v_1(x) = v_0(x) + Ce^{-P(x)}$, où $v_0(x)$ est une solution de $y'(x) + p(x)y(x) = f(x)$.
    \\
    Ce qui est équivalent à $\exists C \in \R: v_1(x) - v_0(x) = Ce^{-P(x)}$
    \\
    \textbf{(2)}
    \\
    Par le principe de \important{superposition des solutions}, la fonction $v_1(x) - v_0(x)$ est une solution de l'équation $y'(x) + p(x)y(x) = f(x)$ est $v(x) = v_0(x) + Ce^{-P(x)}$ où $C \in \R, x \in I$
    \\
    \textbf{(3)}
    \\
    $y'(x) + p(x)y(x) = 0$ est EDVS $\implies $ la solution générale de cette équation est $v(x) = Ce^{-P(x)}$, $C \in \R$ et $P(x)$ est une primitive de $p(x)$ sur $I$.
    \\
    \textbf{(4)}
    \\
    Donc, par la définition $v(x)$ est la \important{solution générale}.
\end{parag}


